Projekt zaliczeniowy na potrzeby przedmiotu Zarządzanie przedsięwzięciami informatycznymi,
\begin{DoxyEnumerate}
\item Utwórz repozytorium w serwisie github o nazwie \textquotesingle{}nazwisko\+\_\+imie\textquotesingle{} (bez polskich znaków).
\item Repozytorium w chwili utworzenia ma zawierać plik licencji oraz README.\+md z krótkim opisem projektu (np. Projekt zaliczeniowy na potrzeby przedmiotu Zarządzanie przedsięwzięciami informatycznymi, jakieś punkty, itp.).
\item Zapisać zmiany w repozytorium, nadać etykietę \textquotesingle{}init\textquotesingle{}.
\item Dodać dwa pliki z rozszerzeniem $\ast$.h i dwa pliki z rozszerzeniem $\ast$.cpp zawierającymi deklaracje i definicje klas \textquotesingle{}\mbox{\hyperlink{class_sequence}{Sequence}}\textquotesingle{} i \textquotesingle{}\mbox{\hyperlink{class_statistics}{Statistics}}\textquotesingle{}. Klasy muszą zawierać\+: konstruktor domyślny, konstruktor kopiujący, destruktor, funkcje dostępu do składowych, funkcje obliczeniowe. Wprowadzić zmiany do repozytorium i nadać im etykietę \textquotesingle{}class\textquotesingle{}.
\item Napisać dokumentację do klas. W dokumentacji funkcji muszą pojawić się skomplikowane wzory matematyczne (np. całki, sumy, itp.), tabele i obrazki/wykresy. Można wykorzystać różnego rodzaju wzory na szeregi, gęstości rozkładów prawdopodobieństwa, itp. Dodać plik konfiguracyjny programu doxygen. Wprowadzić zmiany do repozytorium i nadać im etykietę \textquotesingle{}documentation\textquotesingle{}.
\item Dodać dokumentację w formie dodatkowych stron i zmodyfikowanej strony głównej. Wprowadzić zmiany do repozytorium i nadać im etykietę \textquotesingle{}doxpage\textquotesingle{}.
\item Utworzyć nową gałąź \textquotesingle{}hotfix\textquotesingle{} i wprowadzić do niej deklarację i definicję klasy \textquotesingle{}Error\+Handle\textquotesingle{}. 
\end{DoxyEnumerate}